\title{\LaTeX\ GSS Speaking Notes}
\author{Karl Ahrendsen}
\date{\today}
\documentclass[12pt]{article}

\begin{document}
\maketitle

\section{Intro}
	*Thank introducer*
	
	This is going to be a little bit different than the GSS
	seminars held previously. The goal here is to increase 
	everyone's experience with \LaTeX. Rather than talking
	a whole lot about it, most of the workshop will be 
	giving you the chance to use \LaTeX\ and learn more about
	it. If you don't have any experience with \LaTeX, we 
	hope that by the end of the workshop you would feel
	comfortable creating a document on your own. If you 
	have used \LaTeX\ in the past, we hope that you'll walk
	away with some useful tricks or newfound knowledge.

	Even though we don't want to talk a whole lot, knowing
	a little bit of background information about \LaTeX\
	will help you understand how to best use it, so that's
	what we're going to start with. 
\section{What is \LaTeX?}
\subsection{\TeX}
	Before we start talking about \LaTeX, we have to talk 
	about what it's built on, \TeX. \TeX\ is just a
	typesetting system for computers. Before the wonderful 
	age of computers, if you 
	wanted to have a well formatted document, you had to
	use a printing press. This involved putting tiny metal 
	letters onto a plate in the correct configuration to 
	make your output. This is essentially what \TeX\ does,
	puts tiny little letters in a specific configuration,
	but for a computer. 

	The goals of \TeX\ were to make high quality documents 
	accessible to everyone on all computers, forever.

	This is all great, but you might be sitting there
	thinking, "I don't want to have to tell the computer
	where to put every single letter! That's ridiculous!"
	You're not alone, that's where LaTeX\ comes in.
\subsection{\LaTeX}
	\LaTeX\ is a high-level language that utilizes \TeX\
	to create its output. It essentially puts you in the
	role of administrator that tells a servant to do 
	all the typesetting.

	This is just a fancy way of saying it's a document
	preparation system, similar to others that you might
	know, like Microsoft Word or LibreOffice Writer.
	The big difference is \LaTeX\ uses plain text, while
	Word and Writer are so called ``what you see is what you
	get" (WYSIWYG) editors. Another key difference is 
	that with \LaTeX, the focus is on the content, rather than
	the presentation. 
\section{Why \LaTeX?}
	There are a variety of reasons that you might like to 
	use \LaTeX, and this is just a short sampling of possible
	reasons. 

	It might be necessary for your particular sub-field. For 
	example, I don't think I would be incorrect to say that 
	all high-energy physics documents are produced in \LaTeX\

	It makes writing and typesetting mathematical equations
	significantly easier. Rather than dealing with accidentally
	deleting the entire variable instead of just the subscript, 
	you can just type what you mean. 

	Referencing equations has gotten better in Microsoft Word 
	where this might not be an issue anymore, but \LaTeX\ provides
	an excellent system for cross-referncing not only equations,
	but also section titles and figures.

	We won't get to talk about it today, but because it is 
	written in plain text, it allows use of version control
	systems that are much easier than saving the date at the end
	of the fileName and stashing them all in a folder. 

	The output just always looks so good. This isn't entirely a 
	superfluous property either. I wrote a paper for Dr. Batelaan
	last semester, and he said that the well formatted output
	with equation numbers and figure captions leads him to
	believe that a lot of time and effort went into making the 
	document, and means that the person probably knows what
	they're talking about. 

\section{What You Need to Know}
	The only file that you need to edit is the *.tex files. 
	When you compile your output, a variety of other files will
	be created, but these can be ignored.

	You'll be seeing a lot of ``backslash commands" which 
	describe the content. Remember, this is what makes \LaTeX\
	different, the focus on content over presentation. A funny 
	story: my adviser thinks of these as ``wrappers" so he 
	sometimes refers to \LaTeX\ as the ``condom language."

	The process that you'll follow when writing a \LaTeX\ document
	is this: write, compile, review, repeat. These are the 
	steps you'll take as you're going through the workshop.

	I don't claim to be an expert on \LaTeX, what little I do know
	I've mostly learned from just googling a problem when I come
	across it. I also will frequently reference documents I have
	written previously to recall commands that I need to use when
	writing.

\section{Workshop}
	So now that you know a little bit of background, it's time to
	actually learn how to do it. And like this quote says, ``The
	best way to learn is to do; the worst way to teach is to talk."
	So we'll let you ``do" now. 

	Here are some of the things you'll learn how to do. We picked
	these items because they should be the bare minimum of what you
	would need to know how to do in order to write your own paper.

	We'll leave this slide up here, which might seem confusing
	right now, but as you complete the workshop, it will start 
	to hopefully make more sense.
\end{document}
